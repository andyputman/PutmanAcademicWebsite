\documentclass[11pt]{article}

\title{\vspace{-40pt}The representation theory of $\SL_n(\Z)$}
\author{Andrew Putman}
\date{}

\usepackage{amsmath,amssymb,amsthm,amscd,amsfonts}
\usepackage{mathtools}
\usepackage{epsfig,pinlabel,paralist}
\usepackage[vmargin=1in, hmargin=1.25in]{geometry}
\usepackage[font=small,format=plain,labelfont=bf,up,textfont=it,up]{caption}
\usepackage{type1cm}
\usepackage{calc}
\usepackage{enumerate}
\usepackage{bm}
\usepackage[all,cmtip]{xy}
\usepackage{eucal}
\usepackage{paralist}
\usepackage{lmodern}
\usepackage[T1]{fontenc}
\usepackage{etoolbox}

\usepackage[bookmarks, bookmarksdepth=2, colorlinks=true, linkcolor=blue, citecolor=blue, urlcolor=blue]{hyperref}

\setlength{\parindent}{0pt}
\setlength{\parskip}{\baselineskip}

\apptocmd{\thebibliography}{\raggedright}{}{}

\numberwithin{equation}{section}

\theoremstyle{plain}
\newtheorem{theorem}{Theorem}[section]
\newtheorem{maintheorem}{Theorem}
\newtheorem{maincorollary}[maintheorem]{Corollary}
\newtheorem{proposition}[theorem]{Proposition}
\newtheorem{lemma}[theorem]{Lemma}
\newtheorem*{unnumberedlemma}{Lemma}
\newtheorem{sublemma}[theorem]{Sublemma}
\newtheorem{corollary}[theorem]{Corollary}
\newtheorem{conjecture}[theorem]{Conjecture}
\newtheorem{question}[theorem]{Question}
\newtheorem{problem}[theorem]{Problem}
\newtheorem{claima}{Claim}
\newtheorem*{claim}{Claim}
\newtheorem{step}{Step}
\newtheorem{subcase}{Subcase}
\renewcommand{\themaintheorem}{\Alph{maintheorem}}

\theoremstyle{definition}
\newtheorem{assumption}{Assumption}
\newtheorem{defn}[theorem]{Definition}
\newenvironment{definition}[1][]{\begin{defn}[#1]\pushQED{\qed}}{\popQED \end{defn}}
\newtheorem{notation}[theorem]{Notation}

\theoremstyle{remark}
\newtheorem{rmk}[theorem]{Remark}
\newenvironment{remark}[1][]{\begin{rmk}[#1] \pushQED{\qed}}{\popQED \end{rmk}}
\newtheorem{remarks}[theorem]{Remarks}
\newtheorem{eg}[theorem]{Example}
\newenvironment{example}[1][]{\begin{eg}[#1] \pushQED{\qed}}{\popQED \end{eg}}
\newtheorem{warning}[theorem]{Warning}
\newtheorem{numberedremark}[theorem]{Remark}

% Sets of Functions
\DeclareMathOperator{\Hom}{Hom}
\DeclareMathOperator{\End}{End}
\DeclareMathOperator{\Iso}{Iso}
\DeclareMathOperator{\Diff}{Diff}
\DeclareMathOperator{\Homeo}{Homeo}
\DeclareMathOperator{\Ker}{ker}
\DeclareMathOperator{\Coker}{coker}
\DeclareMathOperator{\Image}{Im}

% My Favorite Groups
\DeclareMathOperator{\Mod}{Mod}
\DeclareMathOperator{\MCG}{MCG}
\newcommand\Torelli{\ensuremath{{\mathcal I}}}
\DeclareMathOperator{\IA}{IA}
\DeclareMathOperator{\Sp}{Sp}
\DeclareMathOperator{\SO}{SO}
\DeclareMathOperator{\GL}{GL}
\DeclareMathOperator{\SL}{SL}
\DeclareMathOperator{\PSL}{PSL}
\DeclareMathOperator{\U}{U}
\DeclareMathOperator{\SU}{SU}
\newcommand\fsl{\ensuremath{\mathfrak{sl}}}
\newcommand\fgl{\ensuremath{\mathfrak{gl}}}
\DeclareMathOperator{\Mat}{Mat}
\DeclareMathOperator{\ad}{ad}

% Important Spaces
\newcommand\HBolic{\ensuremath{\mathbb{H}}}
\newcommand\Teich{\ensuremath{{\mathcal T}}}
\newcommand\CNosep{\ensuremath{\mathcal{CNS}}}
\newcommand\Sphere[1]{\ensuremath{\mathbf{S}^{#1}}}
\newcommand\Ball[1]{\ensuremath{\mathbf{B}^{#1}}}
\newcommand\Proj{\ensuremath{\mathbb{P}}}

% Number Systems
\newcommand\R{\ensuremath{\mathbb{R}}}
\newcommand\C{\ensuremath{\mathbb{C}}}
\newcommand\Z{\ensuremath{\mathbb{Z}}}
\newcommand\Q{\ensuremath{\mathbb{Q}}}
\newcommand\N{\ensuremath{\mathbb{N}}}
\newcommand\Field{\ensuremath{\mathbb{F}}}
\DeclareMathOperator{\Char}{char}

% (Co-)Homology
\DeclareMathOperator{\HH}{H}
\DeclareMathOperator{\CC}{C}
\DeclareMathOperator{\Tor}{Tor}
\newcommand\RH{\ensuremath{\widetilde{\HH}}}

% Misc
\DeclareMathOperator{\Max}{max}
\DeclareMathOperator{\Min}{min}
\DeclareMathOperator{\Aut}{Aut}
\DeclareMathOperator{\Out}{Out}
\DeclareMathOperator{\Inn}{Inn}
\DeclareMathOperator{\Ind}{Ind}
\DeclareMathOperator{\Res}{Res}
\DeclareMathOperator{\Interior}{Int}
\newcommand\Span[1]{\ensuremath{\langle #1 \rangle}}
\newcommand\CaptionSpace{\hspace{0.2in}}
\DeclareMathOperator{\Dim}{dim}
\newcommand\Set[2]{\ensuremath{\{\text{#1 $|$ #2}\}}}
\newcommand\Pres[2]{\ensuremath{\langle \text{#1 $|$ #2} \rangle}}

% Figures
\newcommand\Figure[4]{
\begin{figure}[t]
\centering
\centerline{\psfig{file=#2,scale=#4}}
\caption{#3}
\label{#1}
\end{figure}}

% Document specific macros go here
\newcommand\orho{\ensuremath{\overline{\rho}}}
\newcommand\otau{\ensuremath{\overline{\tau}}}
\newcommand\oa{\ensuremath{\overline{a}}}
\newcommand\cP{\ensuremath{\mathcal{P}}}
\newcommand\cC{\ensuremath{\mathcal{C}}}
\newcommand\cK{\ensuremath{\mathcal{K}}}
\newcommand\cO{\ensuremath{\mathcal{O}}}
\newcommand\og{\ensuremath{\overline{g}}}
\newcommand\cB{\ensuremath{\mathcal{B}}}
\newcommand\cM{\ensuremath{\mathcal{M}}}
\newcommand\uR{\ensuremath{\underline{R}}}
\newcommand\coB{\ensuremath{\overline{\mathcal{B}}}}
\newcommand\bk{\ensuremath{\mathbf{k}}}
\newcommand\tSLn{\ensuremath{\widehat{\SL_n(\Z)}}}

\begin{document}

\vspace{-10pt}
\maketitle

\vspace{-24pt}
\begin{abstract}
\noindent
We give a fairly complete description of the finite-dimensional characteristic $0$ representation theory of $\SL_n(\Z)$ for 
$n \geq 3$, following work of Margulis and Lubotzky.
\end{abstract}

\section{Introduction}

Fix a field $\bk$ of characteristic $0$.  The goal of this note is to describe the finite-dimensional
representations of $\SL_n(\Z)$ over $\bk$.  Since $\SL_2(\Z)$ is very close to a free group,
its representation theory is quite wild and we will have little to say about it.  For $n \geq 3$, however,
there is a beautiful answer whose basic idea goes back to Lubotzky's PhD thesis \cite{LubotzkyThesis}, building
on work of Margulis.

\paragraph{Rational representations.}
There are two natural sources of representations of $\SL_n(\Z)$.  The first are rational
representations of the algebraic group $\SL_n(\bk)$, which can be restricted to the subgroup
$\SL_n(\Z)<\SL_n(\bk)$.  These representations are very well-behaved: they decompose into
direct sums of irreducible representations, and at least when $\bk$ is algebraically closed
these irreducible representations are completely understood (a good starting point for
this is \cite{HumphreysBook}).  As we will prove later, a rational representation $V$ of
$\SL_n(\bk)$ is irreducible if and only if its restriction to $\SL_n(\Z)$ is irreducible.
The key point is that $\SL_n(\Z)$ is Zariski dense in $\SL_n(\bk)$.  It follows that this picture
remains unchanged when the representations are restricted to $\SL_n(\Z)$.

\paragraph{Finite groups.}
The second source of representations of $\SL_n(\Z)$ come from finite quotients.  If
$\pi\colon \SL_n(\Z) \rightarrow F$ is a surjective map from $\SL_n(\Z)$ to a finite group $F$ and
$V$ is a representation of $F$ over $\bk$, then $V$ is also a representation of
$\SL_n(\Z)$ via $\pi$.  For $n \geq 3$, these finite quotients are well-understood:
the congruence subgroup property for $\SL_n(\Z)$ (proved
independently by Bass--Lazard--Serre \cite{BassLazardSerre} and Mennicke \cite{Mennicke}) says that
for every finite quotient $\pi\colon \SL_n(\Z) \rightarrow F$, there is some $\ell \geq 2$ such that
$\pi$ factors through the map
$\SL_n(\Z) \longrightarrow \SL_n(\Z/\ell)$
that reduces matrices modulo $\ell$.  It is thus enough to only consider representations
of $\SL_n(\Z/\ell)$ for $\ell \geq 2$.  These are also well-understood, at least when $\ell$ is prime.

\paragraph{Profinite completion}
The finite quotients of $\SL_n(\Z)$ form an inverse system of finite groups, and their inverse
limit 
\[\tSLn = \varprojlim_{\substack{K \lhd \SL_n(\Z) \\ [\SL_n(\Z):K] < \infty}} \SL_n(\Z)/K\]
is the {\em profinite completion} of $\SL_n(\Z)$.  Endowing each $\SL_n(\Z)/K$ with the discrete topology and
$\tSLn$ with the resulting inverse limit topology, the group $\tSLn$ is a compact totally disconnected
topological group.  It is immediate from the definitions that a finite-dimensional representation
$\rho\colon \SL_n(\Z) \rightarrow \GL(V)$ factors through a finite group if and only if it
factors through a continuous homomorphism $\tSLn \rightarrow \GL(V)$, where $\GL(V)$ is endowed
with the discrete topology.

\begin{remark}
Consider some $\ell \geq 2$.  Letting
\[\ell = p_1^{k_1} \cdots p_m^{k_m},\]
be its prime factorization, the Chinese remainder theorem can be used to prove that
\[\SL_n(\Z/\ell) \cong \SL_n(\Z/p_1^{k_1}) \times \cdots \times \SL_n(\Z/p_m^{k_m}).\]
Using this, the congruence subgroup property implies that
\[\tSLn \cong \prod_p \varprojlim_k \SL_n(\Z/p^k) \cong \prod_p \SL_n(\Z_p).\qedhere\]
\end{remark}

\paragraph{Combining the two families}
The main theorem we will discuss says informally that all representations of 
$\SL_n(\Z)$ are built from combinations of the above basic families of representations.
The natural inclusions
\[\SL_n(\Z) \hookrightarrow \SL_n(\bk) \quad \text{and} \quad \SL_n(\Z) \hookrightarrow \tSLn\]
combine to give an inclusion
\[\SL_n(\Z) \hookrightarrow \SL_n(\bk) \times \tSLn.\]
Define a {\em rational representation} of $\SL_n(\bk) \times \tSLn$ to be a finite-dimensional
$\bk$-vector space $V$ equipped with a homomorphism
$\rho\colon \SL_n(\bk) \times \tSLn \rightarrow \GL(V)$ such that 
$\rho|_{\SL_n(\bk)}$ is rational and $\rho|_{\tSLn}$ is
continuous.  We then have the following theorem, which was essentially proved by Lubotzky \cite{LubotzkyThesis}.

\begin{maintheorem}[Extending representations]
\label{maintheorem:lubotzky}
Let $\bk$ be a field of characteristic $0$ and let $n \geq 3$.  Let $V$ be a finite-dimensional
vector space over $\bk$ and let $\rho\colon \SL_n(\Z) \rightarrow \GL(V)$ be a representation.
Then $\rho$ can be uniquely extended to a rational representation
$\rho\colon \SL_n(\bk) \times \tSLn \rightarrow \GL(V)$.  Moreover, if $W \subset V$ is a subspace that
is an $\SL_n(\Z)$-subrepresentation of $V$, then $W$ is also an
$\SL_n(\bk) \times \tSLn$-subrepresentation.
\end{maintheorem}

\begin{remark}
Theorem \ref{maintheorem:lubotzky} implies that $\SL_n \times \tSLn$ is the {\em proalgebraic completion}
of $\SL_n(\Z)$; see, e.g., \cite{BassLubotzky}.
\end{remark}

\paragraph{Consequences.}
Theorem \ref{maintheorem:lubotzky} implies that the finite-dimensional representations of $\SL_n(\Z)$ over
$\bk$ are in bijection with the rational representations of 
$\SL_n(\bk) \times \tSLn$.  Moreover, this bijection preserves subrepresentations, and thus restricts
to a bijection between the irreducible representations of these groups.  Finally, the fact that
this bijection preserves subrepresentations implies that it takes direct sums of representations
to direct sums of representations.

Now, every rational representation of $\SL_n(\bk) \times \tSLn$ decomposes as a direct sum of irreducible
subrepresentations; indeed, such a representation $\rho\colon \SL_n(\bk) \times \tSLn \rightarrow \GL(V)$
factors through $\SL_n(\bk) \times F$ for some finite group $F$, making $V$ into a rational
representation of the algebraic group $\SL_n(\bk) \times F$.  It is standard that such representations
decompose into direct sums of irreducibles.  Combining this with Theorem \ref{maintheorem:lubotzky}, we deduce
the following.

\begin{maincorollary}[Semisimplicity]
\label{maincorollary:semisimple}
Let $\bk$ be a field of characteristic $0$ and let $n \geq 3$.  Then every finite-dimensional representation
of $\SL_n(\Z)$ over $\bk$ decomposes as a direct sum of irreducible representations.
\end{maincorollary}

\paragraph{Irreducible representations.}
If $\bk$ is algebraically closed, then the irreducible rational representations of $\SL_n(\bk) \times \tSLn$
have a simple description.  Recall that if $G$ and $H$ are finite groups and $\bk$ is an algebraically
closed field of characteristic $0$, then the irreducible representations of $G \times H$ over $\bk$ are precisely
the representations of the form $V \otimes W$, where $V$ is an irreducible representation of $G$ and
$W$ is an irreducible representation of $H$.  The following theorem says that the same thing holds
for $\SL_n(\bk) \times \tSLn$.

\begin{maintheorem}[Irreducible representations]
\label{maintheorem:irreducible}
Let $\bk$ be an algebraically closed field of characteristic $0$ and let $n \geq 2$.  Then the
irreducible rational representations of $\SL_n(\bk) \times \tSLn$ are precisely those of the
form $V \otimes W$, where $V$ and $W$ are as follows:
\begin{compactitem}
\item $V$ is an irreducible finite-dimensional rational representation of $\SL_n(\bk)$.
\item $W$ is an irreducible finite-dimensional continuous representation of $\tSLn$ over $\bk$.
\end{compactitem}
\end{maintheorem}

Theorem \ref{maintheorem:lubotzky} implies for $n \geq 3$ that the restrictions of these to 
$\SL_n(\Z)$ are precisely the irreducible finite-dimensional representations of $\SL_n(\Z)$ over $\bk$.

\paragraph{Super-rigidity.}
The main tool for proving Theorem \ref{maintheorem:lubotzky} is super-rigidity for $\SL_n(\Z)$, one version
of which is as follows.  The use of this theorem is the key place where the assumption that $n \geq 3$ is
used.

\begin{maintheorem}[Super-rigidity]
\label{maintheorem:superrigidity}
Let $\bk$ be a field of characteristic $0$ and let $n \geq 3$.  
Let $V$ be a finite-dimensional $\bk$-vector space and $\rho\colon \SL_n(\Z) \rightarrow \GL(V)$
be a representation.  Then there exists a rational representation $f\colon \SL_n(\bk) \rightarrow \GL(V)$ of
the algebraic group $\SL_n(\bk)$ and a finite-index subgroup $K$ of $\SL_n(\Z)$ such that $f|_K = \rho|_K$.
\end{maintheorem}

The history of Theorem \ref{maintheorem:superrigidity} is a little complicated.
For $\bk = \Q$, it was originally proved by Bass--Milnor--Serre \cite{BassMilnorSerre}
using the congruence subgroup property.  This proof was later extended to general $\bk$ by
Ragunathan \cite{Ragunathan}.  Shortly before this, however, Margulis proved his famous
super-rigidity theorem for higher-rank lattices \cite{MargulisBook}.  One special case of this
result is as follows (see \cite[Theorem 16.1.1]{WitteBook}):
\begin{compactitem}
\item For some $n \geq 3$, let $\Gamma$ be a non-uniform lattice in $\SL_n(\R)$ and let
$\rho\colon \Gamma \rightarrow \GL_m(\R)$ be any homomorphism.  Then there exists a continuous
homomorphism $f\colon \SL_n(\R) \rightarrow \GL_m(\R)$ and a finite-index subgroup $K<\Gamma$
such that $f|_K = \rho|_K$.
\end{compactitem}
On the one hand, this is much more general than Theorem \ref{maintheorem:superrigidity} since it
applies to {\em all} non-uniform lattices, not just $\SL_n(\Z)$.  On the other hand, it
appears to just work for $\bk = \R$ and to only give a continuous extension of $\rho$.  However,
it is not hard to derive the general case of Theorem \ref{maintheorem:superrigidity} from the above statement.
We will not prove Theorem \ref{maintheorem:superrigidity}, but in Appendix \ref{appendix:rigidity} we will
show how to derive it from the above version of Margulis's result.

\begin{remark}
In \cite[Theorem 6]{Steinberg}, Steinberg sketches a remarkably elementary proof
of Theorem \ref{maintheorem:superrigidity}.  Like Bass--Milnor--Serre and
Ragunathan's proofs, it uses the congruence subgroup property, but otherwise
just relies on elementary facts about linear algebra and algebraic groups.  In
fact, he states it in a somewhat different way, but he proves a stronger result
that is essentially equivalent to Theorem \ref{maintheorem:lubotzky}.
\end{remark} 

\paragraph{Outline.}
We begin in \S \ref{section:rigidity} by proving a density result that will be needed for the
uniqueness part of Theorem \ref{maintheorem:lubotzky}.  Next, in \S \ref{section:lubotzky} we will
prove Theorem \ref{maintheorem:lubotzky}.  In \S \ref{section:irreducible} we will prove Theorem 
\ref{maintheorem:irreducible}, and then finally in our Appendix \S \ref{appendix:rigidity} we will
show how to derive Theorem \ref{maintheorem:superrigidity} from Margulis's work.

\paragraph{Acknowledgments.}
I want to thank Nate Harman, who introduced me to this circle of ideas, which plays an
important role in his beautiful paper \cite{Harman}.

\section{Density}
\label{section:rigidity}

This section is devoted to proving the following lemma, which will play an important
role in the proof of Theorem \ref{maintheorem:lubotzky}.

\begin{theorem}[Density]
\label{theorem:bigdensity}
Let $\bk$ be a field of characteristic $0$ and let $n \geq 2$.  Endow
$\SL_n(\bk)$ with the Zariski topology and $\tSLn$ with its usual profinite
topology.  Then $\SL_n(\Z)$ is dense in $\SL_n(\bk) \times \tSLn$.
\end{theorem}

For the proof of Theorem \ref{theorem:bigdensity}, we will need the following.

\begin{lemma}
\label{lemma:density}
Let $\bk$ be a field of characteristic $0$ and let $n \geq 2$.  Let $K$ be a finite-index subgroup of
$\SL_n(\Z)$.  Then $K$ is Zariski dense in $\SL_n(\bk)$.
\end{lemma}
\begin{proof}
Let $G < \SL_n(\bk)$ be the Zariski closure of $K$.  We must prove that $G = \SL_n(\bk)$.  For
distinct $1 \leq i,j \leq n$, let $R_{ij}(\bk)$ be the root subgroup of $\SL_n(\bk)$ consisting
of matrices obtained from the identity by inserting a single nonzero entry at position $(i,j)$.
We thus have $R_{ij}(\bk) \cong \bk$ as additive groups.  Similarly, define the root subgroup
$R_{ij}(\Z) \cong \Z$ of $\SL_n(\Z)$.  The intersection $R_{ij}(\bk) \cap K$ is a finite-index
subgroup of $R_{ij}(\Z)$, and in particular is infinite.  Since a one-variable polynomial can only
have finitely many zeros, the Zariski closure of $R_{ij}(\bk) \cap K$ must be $R_{ij}(\bk)$.  We
deduce that $R_{ij}(\bk) \subset G$ for all distinct $1 \leq i,j \leq n$.  Since $\SL_n(\bk)$
is generated by these root subgroups, we conclude that $G = \SL_n(\bk)$, as desired.
\end{proof}

\begin{proof}[Proof of Theorem \ref{theorem:bigdensity}]
Let $G < \SL_n(\bk) \times \tSLn$ be the closure of $\SL_n(\Z)$.  We must prove that
$G = \SL_n(\bk) \times \tSLn$.  The key to the proof is the following claim.

\begin{claim}
Let $F$ be a finite quotient of $\SL_n(\Z)$ and let
\[\pi\colon \SL_n(\bk) \times \tSLn \rightarrow \SL_n(\bk) \times F\]
be the projection.  Then $\pi(G) = \SL_n(\bk) \times F$.
\end{claim}
\begin{proof}[Proof of claim]
Let $K \lhd \SL_n(\Z)$ be the kernel of the projection to $F$, so $\pi(K) = K \times 1$.  
Lemma \ref{lemma:density} implies that $K$ is Zariski dense in $\SL_n(\bk)$, so we conclude that
\begin{equation}
\label{eqn:firstineq}
\SL_n(\bk) \times 1 = \overline{K \times 1} \subset \pi(G).
\end{equation}
For all $f \in F$, we can find some element of $\SL_n(\Z)$ projecting to $f$, which implies that
there exists some $f' \in \SL_n(\bk)$ such that $(f',f) \in \pi(G)$.  Since $(f',1) \in \pi(G)$, we
deduce that $(1,f) \in \pi(G)$, so
\begin{equation}
\label{eqn:secondineq}
1 \times F \subset \pi(G).
\end{equation}
Inclusions \ref{eqn:firstineq} and \ref{eqn:secondineq} imply the claim.
\end{proof}

We now turn to proving that $G = \SL_n(\bk) \times \tSLn$.  Let 
\[\cK = \Set{$K$}{$K \lhd \SL_n(\Z)$, $[\SL_n(\Z):K] < \infty$}.\]
The group $\tSLn$ is thus the closed subset of
\[\prod_{K \in \cK} \SL_n(\Z)/K\]
consisting of all $(a_K)_{K \in \cK}$ such that for all $K_1,K_2 \in \cK$ with $K_2 < K_1$, the element
$a_{K_2} \in \SL_n(\Z)/K_2$ projects to $a_{K_1} \in \SL_n(\Z)/K_1$ under the projection
$\SL_n(\Z)/K_2 \rightarrow \SL_n(\Z)/K_1$.  
Consider some
\[\lambda = (b,(a_K)_{K \in \cK}) \in \SL_n(\bk) \times \tSLn.\]
Our goal is to show that $\lambda \in G$.  Enumerate $\cK$ as
\[\cK = \{K_1,K_2,\ldots\}.\]
For all $r \geq 1$, let $L = K_1 \cap \cdots \cap K_r \in \cK$.  By the above claim, we can find
some $\lambda_r = (b,(a_K')_{K \in \cK}) \in G$ such that $a_{L}' = a_L$.  Since $L < K_i$ for
all $1 \leq i \leq r$, it follows that $a_{K_i'} = a_{K_i}$ for all $1 \leq i \leq r$.  The
sequence $\{\lambda_i\}_{i=1}^{\infty}$ is therefore a sequence of elements of $G$ that converges to
$\lambda$.  Since $G$ is closed, we conclude that $\lambda \in G$, as desired.
\end{proof}

\section{Extending representations}
\label{section:lubotzky}

We now prove Theorem \ref{maintheorem:lubotzky}.

\begin{proof}[Proof of Theorem \ref{maintheorem:lubotzky}]
We start by recalling what we have to prove.  Let $\bk$ be a field of characteristic $0$ and 
let $n \geq 3$.  Let $V$ be a finite-dimensional vector space over $\bk$ and let 
$\rho\colon \SL_n(\Z) \rightarrow \GL(V)$ be a representation.  Our goal is to prove that 
$\rho$ can be uniquely extended to a rational representation
$\rho\colon \SL_n(\bk) \times \tSLn \rightarrow \GL(V)$.  Moreover, this extension should be such that
if $W \subset V$ is a subspace that is an $\SL_n(\Z)$-subrepresentation of $V$, then $W$ is also an
$\SL_n(\bk) \times \tSLn$-subrepresentation.

The uniqueness of such an extension (if it exists) is an immediate consequence of
Theorem \ref{theorem:bigdensity} (Density), which also implies that such an extension
must preserve all subrepresentations.  All we must do, therefore, is construct
our extension.  As was pointed out by Lubotzky in \cite[p. 680]{LubotzkyThesis}, the argument
for this was originally given by Serre in the special case of $\SL_2$ (see \cite[p. 502]{SerreCongruence}), and
the general case is exactly the same.

Applying Theorem \ref{maintheorem:superrigidity} (Super-rigidity), we get a rational
representation $f\colon \SL_n(\bk) \rightarrow \GL(V)$ and a finite-index subgroup
$K < \SL_n(\Z)$ such that $f|_K = \rho|_K$.  Replacing $K$ by a deeper finite-index
subgroup if necessary, we can assume that $K$ is a normal subgroup of $\SL_n(\Z)$.
To extend $\rho$ to a rational representation of $\SL_n(\bk) \times \tSLn$, it is
enough to construct a homomorphism $g\colon \SL_n(\Z) \rightarrow \GL(V)$
with the following three properties:
\begin{compactitem}
\item[(a)] For all $x \in K$, we have $g(x) = 1$, and thus $g$ factors through the finite group
$\SL_n(\Z)/K$ and induces a continuous representation
$\og\colon \tSLn \rightarrow \GL(V)$.
\item[(b)] For all $x \in \SL_n(\bk)$ and $y \in \SL_n(\Z)$, the elements $f(x)$ and $g(y)$ of $\GL(V)$ commute, so 
$f \times \og$ is a rational representation of $\SL_n(\bk) \times \tSLn$.
\item[(c)] For all $x \in \SL_n(\Z)$, we have $\rho(x) = f(x) g(x)$.
\end{compactitem}
For (c) to hold, we must define
\[g(y) = f(y)^{-1} \rho(y) \in \GL(V) \quad \quad (y \in \SL_n(\Z)).\]
The resulting map $g\colon \SL_n(\Z) \rightarrow \GL(V)$ is a priori only a set map, but we
will prove that it is a homomorphism satisfying (a) and (b) above.  This will be a 3 step process.

\begin{claima}
For all $y \in \SL_n(\Z)$, the image $g(y)$ only depends on the image of $y$ in $\SL_n(\Z)/K$.
\end{claima}

Since $\rho|_K = f|_K$, we have $g(x) = 1$ for all $x \in K$.  However, since we do not yet know
that $g$ is a homomorphism, this is not enough.  So consider some $y \in \SL_n(\Z)$ and $x \in K$.
We must prove that $g(xy) = g(y)$.  To do this, we calculate $\rho(xy)$ in two ways:
\[\rho(xy) = f(xy) g(xy)
= f(x) f(y) g(xy)\]
and
\[\rho(xy) = \rho(x) \rho(y)
= f(x) g(x) f(y) g(y)
= f(x) f(y) g(y).\]
Comparing these, we see that $g(xy) = g(y)$, as desired.

\begin{claima}
Condition (b) holds: for all $x \in \SL_n(\bk)$ and $y \in \SL_n(\Z)$, the elements $f(x)$ and $g(y)$ of $\GL(V)$ commute.
\end{claima}

Fixing some $y \in \SL_n(\Z)$, the set $\Lambda$ of elements of $\GL(V)$ that commute with $g(y)$ is Zariski closed.
We will prove that $f(\SL_n(\bk)) \subset \Lambda$.  Lemma \ref{lemma:density} implies that 
the finite-index subgroup $K < \SL_n(\Z)$
is Zariski dense in $\SL_n(\bk)$, so it is enough to prove that for all $x \in K$ we have $f(x) \in \Lambda$, i.e.\ that
$f(x)$ and $g(y)$ commute.  To do this, we calculate $\rho(yx)$ in two ways:
\[\rho(yx) = f(yx) g(yx)
= f(y) f(x) g(y),\]
where the second equality uses previously proven fact that $g(yx) = g(y)$, and
\[\rho(yx) = \rho(y) \rho(x)
= f(y) g(y) f(x) g(x) 
= f(y) g(y) f(x),\]
where the third equality uses the fact that $g(x) = 1$ since $x \in K$.  Comparing these, we see that
$f(x) g(y) = g(y) f(x)$, as desired.

\begin{claima}
The map $g$ is a homomorphism, and thus by the first claim (a) holds.
\end{claima}

For $y,y' \in \SL_n(\Z)$, we can apply the previous claim to see that
\[\rho(y y') = \rho(y) \rho(y')
= f(y) g(y) f(y') g(y')
= f(y) f(y') g(y) g(y').\]
Since
\[\rho(y y') = f(y y') g(y y') = f(y) f(y') g(y y'),\]
we conclude that $g(y y') = g(y) g(y')$, as desired.
\end{proof}

\section{Classifying the irreducible representations}
\label{section:irreducible}

We now prove Theorem \ref{maintheorem:irreducible}.

\begin{proof}[Proof of Theorem \ref{maintheorem:irreducible}]
We start by recalling what we must prove.  Let $\bk$ be an algebraically closed field of characteristic $0$ and let 
$n \geq 2$.  Consider a rational representation $U$ of $\SL_n(\bk) \times \tSLn$.  Since $U$ is a continuous
representation of the profinite completion $\tSLn$, the action of $\tSLn$ on $U$ factors through the action of
a finite group $F$.  We must prove that $U$ is
irreducible if and only if $U \cong V \otimes W$, where $V$ and $W$ are as follows:
\begin{compactitem}
\item $V$ is an irreducible finite-dimensional rational representation of $\SL_n(\bk)$.
\item $W$ is an irreducible finite-dimensional representation of $F$.
\end{compactitem}
We prove the two directions of this result separately.

\begin{claim}
Assume that $U \cong V \otimes W$, where $V$ and $W$ are as above.  Then $U$ is irreducible.
\end{claim}

In fact, for this claim it is not important that $V$ is a rational representation of $\SL_n(\bk)$, only
that it is irreducible.  By assumption, $V$ and $W$ are simple modules over the group rings
$\bk[\SL_n(\bk)]$ and $\bk[F]$, respectively.  We remark that by $\bk[\SL_n(\bk)]$ we mean simply
the ordinary group ring, not the ring of regular functions on the affine variety $\SL_n(\bk)$.  
Since $\bk$ is algebraically closed, we can apply the Jacobson Density Theorem (see \cite[Theorem 3.2.2]{EtingofBook} or \cite{PutmanDensity})
and see that the resulting ring maps
$\phi\colon \bk[\SL_n(\bk)] \rightarrow \End(V)$ and $\psi\colon \bk[F] \rightarrow \End(W)$ are surjections.
It follows that the ring map
\[\bk[\SL_n(\bk) \times F] \cong \bk[\SL_n(\bk)] \otimes \bk[F] \stackrel{\phi \otimes \psi}{\longrightarrow} \End(V) \otimes \End(W) \cong \End(V \otimes W)\]
is a surjection and thus that $V \otimes W$ is a simple $\bk[\SL_n(\bk) \times F]$-module, as desired.

\begin{claim}
Assume that $U$ is irreducible.  Then $U \cong V \otimes W$, where $V$ and $W$ are as above.
\end{claim}

First regard $U$ as a representation of $F$.  Since $F$ is finite, $U$ decomposes as a direct sum
of isotypic components.  Since the action of $\SL_n(\bk)$ on $U$ commutes with the action of $F$, it
must preserve these isotypic components.  Since $U$ was assumed to be irreducible, it follows
that $U$ must have a single $F$-isotypic component, i.e.\ that $U \cong W^{\oplus m}$ for some
irreducible $F$-representation $W$ and some $m \geq 0$.  Consider the map
\[\Psi\colon \Hom_F(W,U) \otimes W \rightarrow U\]
defined via the formula $\Psi(\rho \otimes \vec{w}) = \rho(\vec{w})$.  Since $U \cong W^{\oplus m}$, this
map is surjective.  Also, since $\bk$ is algebraically closed we can apply Schur's Lemma to see
that 
\[\Hom_F(W,U) = \Hom_F(W,W^{\oplus m}) \cong \bk^m.\]
We deduce that $\Psi$ is a surjective map between vector spaces of the same dimension, so $\Psi$ is an isomorphism.

The commuting actions of $\SL_n(\bk)$ and $F$ on $U$ thus can be transported via $\Psi$ to give commuting
actions of $\SL_n(\bk)$ and $F$ on $\Hom_F(W,U) \otimes W$.  These actions are easily understood:
\begin{compactitem}
\item The group $F$ acts trivially on $\Hom_F(W,U)$ and acts on $W$ as the given representation.  Indeed, for
$f \in F$ and $\rho \in \Hom_F(W,U)$ and $\vec{w} \in W$ we have
\[f \cdot \Psi(\rho \otimes \vec{w}) = f \cdot \rho(\vec{w}) = \rho(f \cdot \vec{w}) = \Psi(\rho \otimes f \cdot \vec{w}).\]
\item The group $\SL_n(\bk)$ acts on $\Hom_F(W,U)$ by postcomposition (via its action on $U$) and acts trivially on
$W$.  Indeed, for $x \in \SL_n(\bk)$ and $\rho \in \Hom_F(W,U)$ and $\vec{w} \in W$ we have
\[x \cdot \Psi(\rho \otimes \vec{w}) = x \cdot \rho(\vec{w}) = (x \cdot \rho)(\vec{w}) = \Psi(x \cdot \rho \otimes \vec{w}).\]
\end{compactitem}
Since $U$ was assumed to be irreducible, it follows that $V:=\Hom_F(W,U)$ must be an irreducible
$\SL_n(\bk)$-module.  What is more, since $U$ was assumed to be a rational $\SL_n(\bk)$-representation and
$V$ can be realized as an $\SL_n(\bk)$-subrepresentation of $U$ (namely, for any nonzero $\vec{w} \in W$ as
the subspace $\Psi(U \times \vec{w})$), it follows that $V$ is a rational representation of
$\SL_n(\bk)$.  The decomposition $U \cong V \otimes W$ is precisely the one we claimed must exist.
\end{proof}

\appendix

\section{Appendix: On super-rigidity}
\label{appendix:rigidity}

In this section, we will show how to derive Theorem \ref{maintheorem:superrigidity} (Super-rigidity) from
a version of Margulis's super-rigidity theorem.  The special case of Margulis's theorem we will start with
will be as follows:

\begin{theorem}[{\cite[Theorem 16.1.1]{WitteBook}}]
\label{theorem:margulis1}
For some $n \geq 3$, let $\rho\colon \SL_n(\Z) \rightarrow \GL_m(\R)$ be any homomorphism.  Then there exists a continuous
homomorphism $f\colon \SL_n(\R) \rightarrow \GL_m(\R)$ and a finite-index subgroup $K<\SL_n(\Z)$
such that $f|_K = \rho|_K$.
\end{theorem}

\begin{remark}
In the above reference, one can in fact replace $\SL_n(\Z)$ with any non-uniform lattice in $\SL_n(\R)$.
There are more general versions of Margulis's theorem for any irreducible lattice in a higher rank Lie group,
but they are far more complicated to state.
\end{remark}

There are two differences between Theorem \ref{theorem:margulis1} and Theorem \ref{maintheorem:superrigidity}:
\begin{compactitem}
\item Theorem \ref{maintheorem:superrigidity} concerns arbitrary fields $\bk$ of characteristic $0$, not just $\R$.
\item In Theorem \ref{maintheorem:superrigidity}, the extended homomorphism 
$f\colon \SL_n(\bk) \rightarrow \GL_m(\bk)$ is a rational representation of the
algebraic group $\SL_n(\bk)$, while in Theorem \ref{theorem:margulis1} the homomorphism
$f\colon \SL_n(\R) \rightarrow \GL_m(\R)$ is just continuous.
\end{compactitem}
We will deal with the second issue first.  The first step is to upgrade $f$ to a
Lie group homomorphism, i.e.\ a homomorphism that is everywhere smooth:

\begin{theorem}
\label{theorem:autosmooth}
Let $f\colon G \rightarrow H$ be a continuous homomorphism between real Lie groups.  Then
$f$ is smooth.
\end{theorem}
\begin{proof}
This is a nontrivial but standard fact about Lie groups, so we will omit the proof.
See \cite[Proposition 2.4.6]{TaoBook} for an accessible account of it.
\end{proof}

We next upgrade $f$ to a rational representation.  Before we explain how to do this,
we review some cautionary examples.

\begin{example}
\label{example:nonalgebraic}
Here are some examples of nonalgebraic Lie group homomorphisms between $\R$-algebraic groups:
\begin{compactitem}
\item The homomorphism $f\colon \GL_n(\R) \rightarrow \R^{\times}$ defined via the formula
$f(x) = \det(x)^{\sqrt{2}}$.  This kind of phenomenon indicates that it is important to restrict
one's self to perfect groups.
\item Issues can still occur for perfect (or even semisimple) groups.  For example, consider
the adjoint representation of $\SL_3(\R)$ obtained via the derivative of the conjugation action:
\[\ad\colon \SL_3(\R) \longrightarrow \GL(\fsl_3(\R)) \cong \GL_8(\R).\]
The kernel of $\ad$ is the center of $\SL_3(\R)$, which is trivial.  It follows that $\ad$ is an isomorphism
onto its image, which is a closed subgroup $G$ of $\GL_8(\R)$.  The inverse
\[\ad^{-1}\colon G \longrightarrow \SL_3(\R)\]
is certainly smooth; however, it is not algebraic since if it was, then we could extend scalars
to $\C$ and deduce that the complexified adjoint representation
\[\ad_{\C}\colon \SL_3(\C) \longrightarrow \GL(\fsl_3(\C)) \cong \GL_8(\C)\]
is an isomorphism, which is false since $\SL_3(\C)$ has a nontrivial center (of order $3$).  See
\cite{LeeWu} for a discussion of this kind of phenomenon.\qedhere
\end{compactitem}
\end{example}

Despite these examples, the following result still holds.

\begin{lemma}
\label{lemma:continuousrational}
Every Lie group homomorphism $f\colon \SL_n(\R) \rightarrow \GL_m(\R)$ is
$\R$-algebraic.  Similarly, every complex Lie group homomorphism
$F\colon \SL_n(\C) \rightarrow \GL_m(\C)$ is $\C$-algebraic.
\end{lemma}
\begin{proof}
We start by reducing to the complex case.  The derivative of a real Lie group
homomorphism $f\colon \SL_n(\R) \rightarrow \GL_m(\R)$ at the identity is a map 
$\fsl_n(\R) \rightarrow \fgl_m(\R)$ of real Lie algebras.
Since $\fsl_n(\R) \otimes \C \cong \fsl_n(\C)$ and $\fgl_m(\R) \otimes \C = \fgl_m(\C)$, we can tensor
this map with $\C$ and get a map $\fsl_n(\C) \rightarrow \fgl_m(\C)$ of complex Lie algebras.  This is the derivative
at the identify of a homomorphism $F\colon \SL_n(\C) \rightarrow \GL_m(\C)$ of complex Lie groups that fits
into a commutative diagram
\[\xymatrix{
\SL_n(\R) \ar[r]^{f} \ar@{^{(}->}[d] & \GL_m(\R) \ar@{^{(}->}[d] \\
\SL_n(\C) \ar[r]^{F} & \GL_m(\C).}\]
To prove that $f$ is a $\R$-algebraic homomorphism, it is enough to prove that $F$ is a $\C$-algebraic homomorphism.

It remains to prove that every complex Lie group homomorphism $F\colon \SL_n(\C) \rightarrow \GL_m(\C)$ is $\C$-algebraic.  Let
\[\Lambda = \Set{$(x,F(x))$}{$x \in \SL_n(\C)$} \subset \SL_n(\C) \times \GL_m(\C)\]
be the graph of $F$.  There are thus two projections
\[\pi_1\colon \Lambda \stackrel{\cong}{\longrightarrow} \SL_n(\C) \quad \text{and} \quad \pi_2\colon \Lambda \longrightarrow \GL_m(\C),\]
and $F$ factors as
\begin{equation}
\label{eqn:factor}
\SL_n(\C) \stackrel{\pi_1^{-1}}{\longrightarrow} \Lambda \stackrel{\pi_2}{\longrightarrow} \GL_m(\C).
\end{equation}
The Lie algebra of the subgroup $\Lambda$ of the $\C$-algebraic group $\SL_n(\C) \times \GL_m(\C)$ is isomorphic
to $\fsl_n(\C)$.  Not all connected Lie subgroups of $\C$-algebraic groups are
algebraic subgroups, and the Lie algebras of the ones that are are called
{\em algebraic Lie algebras}.
A basic result about algebraic groups is that over an algebraically closed field
of characteristic $0$ like $\C$,
all perfect Lie subalgebras are algebraic (see
\cite[Corollary 7.9]{BorelBook}).  Since $\fsl_n(\C)$ is perfect, we conclude that
$\Lambda$ is a $\C$-algebraic subgroup of $\SL_n(\C) \times \GL_m(\C)$.

The projection
$\pi_1\colon \Lambda \rightarrow \SL_n(\C)$ is a bijective algebraic map between algebraic groups.  Bijective
maps between algebraic varieties need not be isomorphisms; however, they are if the target is smooth (or even normal)
and the varieties are defined over an algebraically closed field of characteristic $0$ (see
\cite{StarrMathOverflow}; the key tool here is Zariski's Main Theorem).  Since algebraic groups
are automatically smooth, we conclude that $\pi_1$ is an isomorphism of algebraic varieties.  Its inverse $\pi_1^{-1}$
is thus algebraic.  Since the projection $\pi_2$ is also algebraic, we conclude from \eqref{eqn:factor} that
$F$ is an algebraic map.
\end{proof}

We now prove the complex case of Theorem \ref{maintheorem:superrigidity}:

\begin{theorem}
\label{theorem:margulis2}
For some $n \geq 3$, let $\rho\colon \SL_n(\Z) \rightarrow \GL_m(\C)$ be any homomorphism.  Then there exists a rational representation 
$f\colon \SL_n(\C) \rightarrow \GL_m(\C)$ and a finite-index subgroup $K<\SL_n(\Z)$
such that $f|_K = \rho|_K$.
\end{theorem}

\begin{proof}[Proof of Theorem \ref{theorem:margulis2}]
Forgetting its complex structure, the group $\GL_m(\C)$ is a Zariski closed subgroup
of the $\R$-algebraic group $\GL_{2m}(\R)$.  Applying Theorem \ref{theorem:margulis1}
to the composition of $\rho\colon \SL_n(\Z) \rightarrow \GL_m(\C)$ with the
inclusion $\GL_m(\C) \hookrightarrow \GL_{2m}(\R)$, we get a continuous homomorphism
$f\colon \SL_n(\R) \rightarrow \GL_{2m}(\R)$ and a finite-index subgroup
$K<\SL_n(\Z)$ such that $f|_{K} = \rho|_{K}$.  Combining Theorem \ref{theorem:autosmooth}
and Lemma \ref{lemma:continuousrational}, we see that $f$ is a rational representation
of $\SL_n(\R)$.  Lemma \ref{lemma:density} says that $K$ is Zariski dense
in $\SL_n(\R)$, so since $\rho(K)$ is contained in the $\R$-algebraic subgroup
$\GL_m(\C)$ of $\GL_{2m}(\R)$, we see that the image of $f$ lies in
$\GL_m(\C)$, so we can regard $f$ as a $\R$-algebraic homomorphism
$f\colon \SL_n(\R) \rightarrow \GL_m(\C)$.  The derivative of $f$ at the identity is an
$\R$-linear Lie algebra homomorphism $\fsl_n(\R) \rightarrow \fgl_m(\C)$.  We can factor
this Lie algebra homomorphism through a $\C$-linear Lie algebra homomorphism
\[\fsl_n(\C) = \fsl_n(\R) \otimes \C  \rightarrow \fgl_m(\C),\]
which is the derivative at the identity of a map $F\colon \SL_n(\C) \rightarrow \GL_m(\C)$
of complex Lie algebras that fits into a commutative diagram
\[\xymatrix{
\SL_n(\R) \ar[r]^{f} \ar@{^{(}->}[d] & \GL_m(\C). \\
\SL_n(\C) \ar[ru]^{F} & }\]
Another application of Lemma \ref{lemma:continuousrational} shows that
$F\colon \SL_n(\C) \rightarrow \GL_m(\C)$ is a rational representation, and
$F|_K = f|_K = \rho|_K$, as desired.
\end{proof}

Before we prove our main result, we need one final lemma showing how to recognize
the field of definition of a rational representation.

\begin{lemma}
\label{lemma:fieldofdef}
Let $f\colon \SL_n(\C) \rightarrow \GL_m(\C)$ be a rational representation, let
$K < \SL_n(\Z)$ be a finite-index subgroup, and let
$\bk$ be a subfield of $\C$ such that $f(K) \subset \GL_m(\bk)$.  Then
$f$ is defined over $\bk$, and thus restricts to a rational representation
$\SL_n(\bk) \rightarrow \GL_m(\bk)$.
\end{lemma}
\begin{proof}
Let $\C[\SL_n]$ be the $\C$-algebra of regular functions $\SL_n(\C) \rightarrow \C$ and
let $\bk[\SL_n]$ be the $\bk$-algebra of such regular functions that are
defined over $\bk$, i.e.\ by polynomials in the entries of $\SL_n(\C)$ with
coefficients in $\bk$.  Since $\SL_n$ is an algebraic group defined over $\Q$, we have
\begin{equation}
\label{eqn:tensorup}
\C[\SL_n] = \bk[\SL_n] \otimes_{\bk} \C.
\end{equation}
Letting $h \in \C[\SL_n]$ be one of the matrix coefficients of $f$, our goal
is to show that $h \in \bk[\SL_n]$.

Since $\C$ is an algebraically closed field of characteristic $0$, the field
$\bk$ is precisely the set of elements of $\C$ that are invariant under all elements
of $\Aut(\C/\bk)$.  Combining this with \eqref{eqn:tensorup}, we see
that $\bk[\SL_n]$ is the set of all elements of $\C[\SL_n]$ that
are invariant under all elements of $\Aut(\C/\bk)$.  Considering some
$\lambda \in \Aut(\C/\bk)$, we see that our goal is equivalent to showing
that $\lambda \circ h = h$.

By assumption, $\lambda \circ h$ and $h$ agree on all elements of $K$.
Lemma \ref{lemma:density} implies that $K$ is Zariski dense in $\SL_n(\C)$,
so this implies that in fact $\lambda \circ h = h$, as desired.
\end{proof}

We finally prove Theorem \ref{maintheorem:superrigidity}.

\begin{proof}[Proof of Theorem \ref{maintheorem:superrigidity}]
We start by recalling what we must prove.
Let $\bk$ be a field of characteristic $0$ and let $n \geq 3$.
For some $m$, let $\rho\colon \SL_n(\Z) \rightarrow \GL_m(\bk)$
be a representation.  We must prove that there exists a rational representation 
$f\colon \SL_n(\bk) \rightarrow \GL_m(\bk)$ of
the algebraic group $\SL_n(\bk)$ and a finite-index subgroup $K$ of $\SL_n(\Z)$ such that $f|_K = \rho|_K$.

Since $\SL_n(\Z)$ is a finitely generated group, there exists a subfield $\bk'$ of $\bk$ with
the following two properties:
\begin{compactitem}
\item[(a)] For all $x \in \SL_n(\Z)$, the matrix entries of $\rho(x)$ lie in $\bk'$.
\item[(b)] The field $\bk'$ is a finitely generated $\Q$-algebra.
\end{compactitem}
By (a), there exists a homomorphism $\rho'\colon \SL_n(\Z) \rightarrow \GL_m(\bk')$ such
that $\rho$ factors as
\[\SL_n(\Z) \stackrel{\rho'}{\longrightarrow} \GL_m(\bk') \hookrightarrow \GL_m(\bk).\]
Since $\bk'$ is a finitely generated $\Q$-algebra, it is isomorphic to a subfield of
$\C$.  Identifying $\bk'$ with this subfield of $\C$ allows us to identify the image
of $\rho'$ with a subfield of $\C$.  We can thus apply Theorem \ref{theorem:margulis2}
and obtain a rational representation $f'\colon \SL_n(\C) \rightarrow \GL_m(\C)$ and
a finite-index subgroup $K<\SL_n(\Z)$ such that $f'|_{K} = \rho'|_{K}$.  Since
\[f'(K) = \rho'(K) \subset \GL_m(\bk') \subset \GL_m(\C),\]
we can apply Lemma \ref{lemma:fieldofdef} to deduce that $f'$ is defined over
$\bk'$, and thus restricts to a rational representation 
$f''\colon \SL_n(\bk') \rightarrow \GL_m(\bk')$.  Extending scalars from $\bk'$
to $\bk$, we obtain a rational representation $f\colon \SL_n(\bk) \rightarrow \GL_m(\bk)$
such that 
\[f|_K = f'|_K = \rho'|_K = \rho|_K,\]
as desired.
\end{proof}

\begin{thebibliography}{CF}
\begin{footnotesize}
\setlength{\itemsep}{-1mm}

\bibitem{BassLazardSerre}
H. Bass, M. Lazard\ and\ J.-P. Serre, Sous-groupes d'indice fini dans ${\bf SL}(n,\,{\bf Z})$, Bull. Amer. Math. Soc. 70 (1964), 385--392.

\bibitem{BassLubotzky}
H. Bass\ et al., The proalgebraic completion of rigid groups, Geom. Dedicata 95 (2002), 19--58.

\bibitem{BassMilnorSerre}
H. Bass, J. Milnor\ and\ J.-P. Serre, Solution of the congruence subgroup problem for ${\rm SL}\sb{n}\,(n\geq 3)$ and ${\rm Sp}\sb{2n}\,(n\geq 2)$, Inst. Hautes \'{E}tudes Sci. Publ. Math. No. 33 (1967), 59--137. 

\bibitem{BorelBook}
A. Borel, {\it Linear algebraic groups}, second edition, Graduate Texts in Mathematics, 126, Springer-Verlag, New York, 1991. 

\bibitem{EtingofBook}
P. Etingof\ et al., {\it Introduction to representation theory}, Student Mathematical Library, 59, American Mathematical Society, Providence, RI, 2011.

\bibitem{Harman}
N. Harman, Effective and infinite-rank superrigidity in the context of representation stability, preprint 2019.

\bibitem{HumphreysBook}
J. E. Humphreys, {\it Linear algebraic groups}, Springer-Verlag, New York, 1975.

\bibitem{LeeWu}
D. H. Lee\ and\ T. S. Wu, Rationality of representations of linear Lie groups, Proc. Amer. Math. Soc. 114 (1992), no.~3, 847--855. 

\bibitem{LubotzkyThesis}
A. Lubotzky, Tannaka duality for discrete groups, Amer. J. Math. 102 (1980), no.~4, 663--689. 

\bibitem{MargulisBook}
G. A. Margulis, {\it Discrete subgroups of semisimple Lie groups}, Ergebnisse der Mathematik und ihrer Grenzgebiete (3), 17, Springer-Verlag, Berlin, 1991.

\bibitem{Mennicke}
J. L. Mennicke, Finite factor groups of the unimodular group, Ann. of Math. (2) 81 (1965), 31--37.

\bibitem{PutmanDensity}
A. Putman, The Jacobson density theorem, informal note.

\bibitem{WitteBook}
D. W. Morris, {\it Introduction to arithmetic groups}, Deductive Press, 2015.

\bibitem{Ragunathan}
M. S. Raghunathan, On the congruence subgroup problem, Inst. Hautes \'{E}tudes Sci. Publ. Math. No. 46 (1976), 107--161. 

\bibitem{SerreCongruence}
J.-P. Serre, Le probl\`eme des groupes de congruence pour $\SL_2$, Ann. of Math. (2) 92 (1970), 489--527.

\bibitem{StarrMathOverflow}
Jason Starr (https://mathoverflow.net/users/13265/jason-starr), Bijection implies isomorphism for algebraic varieties, URL (version: 2017-03-09): https://mathoverflow.net/q/264216

\bibitem{Steinberg}
R. Steinberg, Some consequences of the elementary relations in ${\rm SL}_n$, in {\it Finite groups---coming of age (Montreal, Que., 1982)}, 335--350, Contemp. Math., 45, Amer. Math. Soc., Providence, RI.

\bibitem{TaoBook}
T. Tao, {\it Hilbert's fifth problem and related topics}, Graduate Studies in Mathematics, 153, American Mathematical Society, Providence, RI, 2014.

\end{footnotesize}
\end{thebibliography} 

\begin{footnotesize}
\noindent
\begin{tabular*}{\linewidth}[t]{@{}p{\widthof{Department of Mathematics}+0.5in}@{}p{\linewidth - \widthof{Department of Mathematics} - 0.5in}@{}}
{\raggedright
Andrew Putman\\
Department of Mathematics\\
University of Notre Dame \\
164 Hurley Hall\\
Notre Dame, IN 46556\\
{\tt andyp@nd.edu}}
&
\end{tabular*}\hfill
\end{footnotesize}

\end{document}
