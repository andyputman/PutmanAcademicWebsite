\documentclass[11pt]{article}

\title{\vspace{-50pt}Burnside's $p^a q^b$-theorem}
\author{Andrew Putman\vspace{-6pt}}
\date{}

\usepackage{amsmath,amssymb,amsthm,amscd,amsfonts}
\usepackage{mathtools}
\usepackage{epsfig,pinlabel,paralist}
\usepackage[vmargin=1in, hmargin=1.25in]{geometry}
\usepackage[font=small,format=plain,labelfont=bf,up,textfont=it,up]{caption}
\usepackage{type1cm}
\usepackage{calc}
\usepackage{enumerate}
\usepackage{bm}
\usepackage[all,cmtip]{xy}
\usepackage{eucal}
\usepackage{paralist}
\usepackage{lmodern}
\usepackage[T1]{fontenc}
\usepackage{etoolbox}

\usepackage[bookmarks, bookmarksdepth=2, colorlinks=true, linkcolor=blue, citecolor=blue, urlcolor=blue]{hyperref}

\apptocmd{\thebibliography}{\raggedright}{}{}

\numberwithin{equation}{section}

\theoremstyle{plain}
\newtheorem{theorem}{Theorem}[section]
\newtheorem*{rochlin}{Rochlin's Theorem}
\newtheorem*{rochlin2}{Rochlin's Theorem'}
\newtheorem{maintheorem}{Theorem}
\newtheorem{proposition}[theorem]{Proposition}
\newtheorem{lemma}[theorem]{Lemma}
\newtheorem*{unnumberedlemma}{Lemma}
\newtheorem{sublemma}[theorem]{Sublemma}
\newtheorem{corollary}[theorem]{Corollary}
\newtheorem{conjecture}[theorem]{Conjecture}
\newtheorem{question}[theorem]{Question}
\newtheorem{problem}[theorem]{Problem}
\newtheorem*{claim}{Claim}
\newtheorem{claims}{Claim}
\newtheorem{stepa}{Step}
\newtheorem{stepb}{Step}
\newtheorem{stepc}{Step}
\newtheorem{case}{Case}
\newtheorem{subcase}{Subcase}
\renewcommand{\themaintheorem}{\Alph{maintheorem}}
\newcommand\BeginClaims{\setcounter{claims}{0}}
\newcommand\BeginCases{\setcounter{case}{0}}
\newcommand\BeginSubcases{\setcounter{subcase}{0}}

\theoremstyle{definition}
\newtheorem*{example}{Example}
\newtheorem{assumption}{Assumption}
\newtheorem{defn}[theorem]{Definition}
\newenvironment{definition}[1][]{\begin{defn}[#1]\pushQED{\qed}}{\popQED \end{defn}}
\newtheorem{notation}[theorem]{Notation}

\newtheorem*{remark}{Remark}

% Sets of Functions
\DeclareMathOperator{\Hom}{Hom}
\DeclareMathOperator{\Iso}{Iso}
\DeclareMathOperator{\Diff}{Diff}
\DeclareMathOperator{\Homeo}{Homeo}
\DeclareMathOperator{\Ker}{ker}
\DeclareMathOperator{\Coker}{coker}
\DeclareMathOperator{\Image}{Im}

% My Favorite Groups
\DeclareMathOperator{\Mod}{Mod}
\DeclareMathOperator{\MCG}{MCG}
\newcommand\Torelli{\ensuremath{{\mathcal I}}}
\DeclareMathOperator{\IA}{IA}
\DeclareMathOperator{\Sp}{Sp}
\DeclareMathOperator{\GL}{GL}
\DeclareMathOperator{\SL}{SL}
\DeclareMathOperator{\SO}{SO}
\DeclareMathOperator{\PSL}{PSL}
\DeclareMathOperator{\U}{U}
\DeclareMathOperator{\SU}{SU}
\newcommand\SLLie{\ensuremath{\mathfrak{sl}}}
\newcommand\SpLie{\ensuremath{\mathfrak{sp}}}
\newcommand\GLLie{\ensuremath{\mathfrak{gl}}}
\DeclareMathOperator{\Mat}{Mat}
\DeclareMathOperator{\End}{End}

% Important Spaces
\newcommand\HBolic{\ensuremath{\mathbb{H}}}
\newcommand\Teich{\ensuremath{{\mathcal T}}}
\newcommand\CNosep{\ensuremath{\mathcal{CNS}}}
\newcommand\Sphere[1]{\ensuremath{\mathbf{S}^{#1}}}
\newcommand\Ball[1]{\ensuremath{\mathbf{B}^{#1}}}
\newcommand\Proj{\ensuremath{\mathbb{P}}}

% Number Systems
\newcommand\R{\ensuremath{\mathbb{R}}}
\newcommand\C{\ensuremath{\mathbb{C}}}
\newcommand\Z{\ensuremath{\mathbb{Z}}}
\newcommand\Q{\ensuremath{\mathbb{Q}}}
\newcommand\N{\ensuremath{\mathbb{N}}}
\newcommand\Field{\ensuremath{\mathbb{F}}}
\DeclareMathOperator{\Char}{char}

% (Co-)Homology
\DeclareMathOperator{\HH}{H}
\newcommand\RH{\ensuremath{\widetilde{\HH}}}
\newcommand\Chain{\ensuremath{{\rm C}}}

% Misc
\DeclareMathOperator{\Max}{max}
\DeclareMathOperator{\Min}{min}
\DeclareMathOperator{\Th}{th}
\DeclareMathOperator{\Aut}{Aut}
\DeclareMathOperator{\Out}{Out}
\DeclareMathOperator{\Interior}{Int}
\newcommand\Span[1]{\ensuremath{\langle #1 \rangle}}
\newcommand\CaptionSpace{\hspace{0.2in}}
\DeclareMathOperator{\Dim}{dim}

% Figures
\newcommand\Figure[4]{
\begin{figure}[t]
\centering
\centerline{\psfig{file=#2,scale=#4}}
\caption{#3}
\label{#1}
\end{figure}}

% Document specific macros go here
\DeclareMathOperator{\Gal}{Gal}
\DeclareMathOperator{\cl}{cl}
\newcommand\bk{\ensuremath{\mathbf{k}}}

\begin{document}

\maketitle

\begin{abstract}
We prove Burnside's theorem saying that a group of order $p^a q^b$ for primes $p$ and $q$ is solvable.
\end{abstract}

In this note, we discuss the proof of the following theorem of Burnside \cite{Burnside}.

\begin{maintheorem}
\label{theorem:burnside}
Let $G$ be a group with $|G| = p^a q^b$ for primes $p$ and $q$.  Then $G$ is solvable.
\end{maintheorem}

\noindent
The key to the proof is showing that such a group must contain a nontrivial normal subgroup.
The subgroups we will construct will be of the following form.  All representations in
this note are finite-dimensional and defined over $\C$.

\begin{definition}
Let $V$ be a representation of a group $G$.  The {\em $V$-central subgroup} of $G$, denoted
$Z(V)$, is the subgroup consisting of all $g \in G$ that act on $V$ by homotheties, i.e.\ such that
there exists some $\lambda \in \C$ such that $g \cdot \vec{v} = \lambda \vec{v}$ for all
$\vec{v} \in V$.
\end{definition}

\begin{remark}
It is clear that $Z(V)$ is a normal subgroup of $G$; indeed, for $g \in Z(V)$ and $\lambda$ as above, for
all $h \in G$ we have
\[h^{-1} g h \cdot \vec{v} = h^{-1} \cdot (\lambda (h \cdot \vec{v})) = \lambda (h^{-1} h) \cdot \vec{v} = \lambda \vec{v} \quad \quad \text{for all $\vec{v} \in V$}.\]
so $h^{-1} g h \in Z(G)$.
\end{remark}

How can we detect nontrivial $Z(V)$?  With respect to an appropriate basis, the matrices
representing the action of elements of $G$ on $V$ are algebraic integers (see, e.g., 
\cite{PutmanAlg}).  It follows that for
$g \in Z(V)$, the action of $g$ on $V$ is a homothety with scaling factor an algebraic integer.  Letting
$\chi$ be the character of $V$, this scaling factor is precisely $\chi(g)/n$, where $n = \dim(V)$.  This must
therefore be an algebraic integer.  The following lemma is a converse to this:

\begin{lemma}
\label{lemma:detectcentral}
Let $G$ be a finite group and let $V$ be an $n$-dimensional representation of $G$ with character $\chi$.  For
some $g \in G$, assume that $\chi(g)/n$ is a nonzero algebraic integer.  Then $g \in Z(V)$.
\end{lemma}
\begin{proof}
Let $\lambda_1,\ldots,\lambda_n \in \C$ be the eigenvalues of the action of $g$ on $V$.  Our goal is to
prove that the $\lambda_i$ are all equal.  Assume otherwise.  Each of the $\lambda_i$ is a root of unity,
so this implies that
\[|\chi(g)| = |\lambda_1+\cdots+\lambda_n| < n.\]
This can be improved as follows.  Let $\bk/\Q$ be a Galois extension containing each of the $\lambda_i$.  For
each $\phi \in \Gal(\bk/\Q)$, the elements $\phi(\lambda_1),\ldots,\phi(\lambda_n) \in \C$ are also roots
of unity that are not all equal, so we have
\[|\phi(\chi(g))| = |\phi(\lambda_1)+\cdots+\phi(\lambda_n)| < n.\]
We thus see that
\[\prod_{\phi \in \Gal(\bk/\Q)} |\phi\left(\frac{\chi(g)}{n}\right)| < 1.\]
Since $\chi(g)/n$ is a nonzero algebraic integer contained in $\bk$, the left hand side lies in $\Z$, and
thus must be $0$, a contradiction.
\end{proof}

The following technical lemma will help us verify the hypotheses of Lemma \ref{lemma:detectcentral}.  For
$g \in G$, let $\cl(g)$ denote the conjugacy class of $g$.

\begin{lemma}
\label{lemma:conjugacymult}
Let $G$ be a finite group and let $V$ be an $n$-dimensional irreducible representation of $G$ with
character $\chi$.  For all $g \in G$, the number $|\cl(g)| \chi(g)/n$ is an algebraic integer.
\end{lemma}
\begin{proof}
Fix some $g \in G$, and define
\[\omega = \sum_{h \in \cl(g)} h \in \C[G].\]
The element $\omega$ lies in the center of the ring $\C[G]$, so it acts on $V$ by $\C[G]$-module endomorphisms.
Since $V$ is irreducible, Schur's Lemma says that $\End_{\C[G]}(V) = \C$, so there exists some
$\lambda \in \C$ such that $\omega \cdot \vec{v} = \lambda \vec{v}$ for all $\vec{v} \in V$.  Taking traces,
we see that
\[n \lambda = \chi(\omega) = \sum_{h \in \cl(g)} \chi(h) = |\cl(g)| \chi(g),\]
so
\[\lambda = |\cl(g)| \chi(g)/n.\]
With respect to an appropriate basis, the entries of the matrices representing the action of $G$ on $V$ are
algebraic integers (see, e.g., \cite{PutmanAlg}).  It follows that $\lambda$ is an algebraic integer, and the lemma follows.
\end{proof}

We will use these results to prove the following key proposition:

\begin{proposition}
\label{proposition:findcentral}
Let $G$ be a finite group such that there exists some $g \in G$ with $|\cl(g)| = p^k$ for some prime $p$ and
some $k \geq 1$.  Then there exists some nontrivial representation $V$ of $G$ such that $Z(V) \neq 1$.
\end{proposition}
\begin{proof}
Let $V_1,\ldots,V_r$ be the irreducible representations of $G$, ordered such that $V_1$ is the trivial
representation.  For $1 \leq i \leq r$, let $\chi_i$ be the character of $V_i$ and $n_i = \dim(V_i)$.
Since $|\cl(g)| \neq 1$, we have $g \neq 1$, so the orthogonality of the columns of the character table says
that
\[0 = \sum_{i=1}^r \chi_i(g) \overline{\chi_i(1)} = \sum_{i=1}^r n_i \chi_i(g) = 1 + \sum_{i=2}^r n_i \chi_i(g).\]
Thus
\[\frac{-1}{p} = \sum_{i=2}^r \frac{n_i \chi_i(g)}{p}.\]
From this, we see that there must exist some $2 \leq j \leq r$ such that $n_j \chi_j(g)/p$ is not
an algebraic integer.  Since $\chi_j(g)$ is a sum of roots of unity, it is an algebraic integer.  We
deduce that $p$ does not divide $n_j$ and that $\chi_j(g) \neq 0$.  Sine $|\cl(g)| = p^k$ and $n_j$
are coprime, we can find $a,b \in \Z$ such that
\[1 = a |\cl(g)| + b n_j.\]
Multiplying this by $\chi_j(g)/n_j$, we get that
\[\frac{\chi_j(g)}{n_j} = a \frac{|\cl(g)| \chi_j(g)}{n_j} + b \chi_j(g).\]
Lemma \ref{lemma:conjugacymult} implies that this first term is an algebraic integer, and since $\chi_j(g)$
is an algebraic integer the second term is as well.  We conclude that $\chi_j(g)/n_j$ is an algebraic
integer, so by Lemma \ref{lemma:detectcentral} we have that $g \in Z(V_j)$, as desired.
\end{proof}

\begin{proof}[Proof of Theorem \ref{theorem:burnside}]
Consider a group $G$ with $|G| = p^a q^b$ for primes $p$ and $q$.  Our goal is to prove that $G$
is solvable.  By induction, we can assume that this holds for all such groups of smaller order.  We
can also assume that $G$ is nonabelian since abelian groups are trivially solvable.  Finally, we can
assume that $p$ and $q$ are distinct and that $a,b \geq 1$ since otherwise $G$ has prime power order and
hence is nilpotent.  It is enough to prove that under these circumstances, there exists a normal subgroup
$N \lhd G$ that is nontrivial in the sense that $1 \subsetneq N \subsetneq G$.  Indeed,
our inductive hypothesis will then imply that both $N$ and $G/N$ are solvable, so $G$ is as well.

Let $H<G$ be a Sylow $q$-subgroup.  Since $|H| = q^b$, the group $H$ is a $q$-group and thus is nilpotent.  In
particular, its center $Z(H)$ satisfies $Z(H) \neq 1$.  Let $g \in Z(H)$ be nontrivial.  The centralizer $C_G(g)$ thus
contains $H$, so 
\[|\cl(g)| = |G|/|C_G(g)| = p^k \quad \quad \text{for some $k \leq a$}.\]  
If $k=0$, then $g \in Z(G)$, so $Z(G)$ is our desired
nontrivial normal subgroup.  If $k \geq 1$, then instead Proposition \ref{proposition:findcentral} says that
there exists a nontrivial representation $V$ of $G$ such that $Z(V) \neq 1$.  If $V$ is not a faithful
representation of $G$, then its kernel is the desired nontrivial normal subgroup.  If $V$ is faithful, then
$Z(V) \neq G$ since $G$ is nonabelian, so $Z(V)$ is the desired nontrivial normal subgroup.
\end{proof}

\begin{thebibliography}{HH}
\begin{footnotesize}
\setlength{\itemsep}{-1mm}

\bibitem{Burnside}
W. Burnside, On Groups of Order $p^{\alpha} q^{\beta}$, Proc. London Math. Soc. (2) 1 (1904), 388--392.

\bibitem{PutmanAlg}
A. Putman, Algebraicity of matrix entries of representations, informal note.
 
\end{footnotesize}
\end{thebibliography}

\begin{footnotesize}
\noindent
\begin{tabular*}{\linewidth}[t]{@{}p{\linewidth - \widthof{Department of Mathematics}}@{}p{\widthof{Department of Mathematics}}@{}}
&{\raggedright
Andrew Putman\\
Department of Mathematics\\
University of Notre Dame \\
255 Hurley Hall\\
Notre Dame, IN 46556\\
{\tt andyp@nd.edu}}
\end{tabular*}\hfill
\end{footnotesize}


\end{document}

