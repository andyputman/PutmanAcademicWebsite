\documentclass[11pt]{article}

\title{\vspace{-50pt}Representations of products}
\author{Andrew Putman\vspace{-6pt}}
\date{}

\usepackage{amsmath,amssymb,amsthm,amscd,amsfonts}
\usepackage{mathtools}
\usepackage{epsfig,pinlabel,paralist}
\usepackage[vmargin=1in, hmargin=1.25in]{geometry}
\usepackage[font=small,format=plain,labelfont=bf,up,textfont=it,up]{caption}
\usepackage{type1cm}
\usepackage{calc}
\usepackage{enumerate}
\usepackage{bm}
\usepackage[all,cmtip]{xy}
\usepackage{eucal}
\usepackage{paralist}
\usepackage{lmodern}
\usepackage[T1]{fontenc}
\usepackage{etoolbox}

\usepackage[bookmarks, bookmarksdepth=2, colorlinks=true, linkcolor=blue, citecolor=blue, urlcolor=blue]{hyperref}

\apptocmd{\thebibliography}{\raggedright}{}{}

\newtheorem{claimx}{Claim}
\newenvironment{claim}[1]
 {\renewcommand\theclaimx{#1}\claimx}
 {\endclaimx}

\newtheorem{stepx}{Step}
\newenvironment{step}[1]
 {\renewcommand\thestepx{#1}\stepx}
 {\endstepx}

\newtheorem{substepx}{Substep}
\newenvironment{substep}[1]
 {\renewcommand\thesubstepx{#1}\substepx}
 {\endsubstepx}

\numberwithin{equation}{section}

\theoremstyle{plain}
\newtheorem{theorem}{Theorem}[section]
\newtheorem*{rochlin}{Rochlin's Theorem}
\newtheorem*{rochlin2}{Rochlin's Theorem'}
\newtheorem{maintheorem}{Theorem}
\newtheorem{proposition}[theorem]{Proposition}
\newtheorem{lemma}[theorem]{Lemma}
\newtheorem*{unnumberedlemma}{Lemma}
\newtheorem{sublemma}[theorem]{Sublemma}
\newtheorem{corollary}[theorem]{Corollary}
\newtheorem{conjecture}[theorem]{Conjecture}
\newtheorem{question}[theorem]{Question}
\newtheorem{problem}[theorem]{Problem}
\renewcommand{\themaintheorem}{\Alph{maintheorem}}

\theoremstyle{definition}
\newtheorem*{example}{Example}
\newtheorem{assumption}{Assumption}
\newtheorem{defn}[theorem]{Definition}
\newenvironment{definition}[1][]{\begin{defn}[#1]\pushQED{\qed}}{\popQED \end{defn}}
\newtheorem{notation}[theorem]{Notation}

\newtheorem*{remark}{Remark}

% Sets of Functions
\DeclareMathOperator{\Hom}{Hom}
\DeclareMathOperator{\Iso}{Iso}
\DeclareMathOperator{\Diff}{Diff}
\DeclareMathOperator{\Homeo}{Homeo}
\DeclareMathOperator{\Ker}{ker}
\DeclareMathOperator{\Coker}{coker}
\DeclareMathOperator{\Image}{Im}

% My Favorite Groups
\DeclareMathOperator{\Mod}{Mod}
\DeclareMathOperator{\MCG}{MCG}
\newcommand\Torelli{\ensuremath{{\mathcal I}}}
\DeclareMathOperator{\IA}{IA}
\DeclareMathOperator{\Sp}{Sp}
\DeclareMathOperator{\GL}{GL}
\DeclareMathOperator{\SL}{SL}
\DeclareMathOperator{\SO}{SO}
\DeclareMathOperator{\PSL}{PSL}
\DeclareMathOperator{\U}{U}
\DeclareMathOperator{\SU}{SU}
\newcommand\SLLie{\ensuremath{\mathfrak{sl}}}
\newcommand\SpLie{\ensuremath{\mathfrak{sp}}}
\newcommand\GLLie{\ensuremath{\mathfrak{gl}}}
\DeclareMathOperator{\Mat}{Mat}
\DeclareMathOperator{\End}{End}

% Important Spaces
\newcommand\HBolic{\ensuremath{\mathbb{H}}}
\newcommand\Teich{\ensuremath{{\mathcal T}}}
\newcommand\CNosep{\ensuremath{\mathcal{CNS}}}
\newcommand\Sphere[1]{\ensuremath{\mathbf{S}^{#1}}}
\newcommand\Ball[1]{\ensuremath{\mathbf{B}^{#1}}}
\newcommand\Proj{\ensuremath{\mathbb{P}}}
\DeclareMathOperator{\Gr}{Gr}

% Number Systems
\newcommand\R{\ensuremath{\mathbb{R}}}
\newcommand\C{\ensuremath{\mathbb{C}}}
\newcommand\Z{\ensuremath{\mathbb{Z}}}
\newcommand\Q{\ensuremath{\mathbb{Q}}}
\newcommand\N{\ensuremath{\mathbb{N}}}
\newcommand\Field{\ensuremath{\mathbb{F}}}
\DeclareMathOperator{\Char}{char}

% (Co-)Homology
\DeclareMathOperator{\HH}{H}
\newcommand\RH{\ensuremath{\widetilde{\HH}}}
\newcommand\Chain{\ensuremath{{\rm C}}}

% Misc
\DeclareMathOperator{\Max}{max}
\DeclareMathOperator{\Min}{min}
\DeclareMathOperator{\Th}{th}
\DeclareMathOperator{\Aut}{Aut}
\DeclareMathOperator{\Out}{Out}
\DeclareMathOperator{\Interior}{Int}
\newcommand\Span[1]{\ensuremath{\langle #1 \rangle}}
\newcommand\CaptionSpace{\hspace{0.2in}}
\DeclareMathOperator{\Dim}{dim}
\newcommand\Set[2]{\ensuremath{\{\text{#1 $|$ #2}\}}}

% Figures
\newcommand\Figure[4]{
\begin{figure}[t]
\centering
\centerline{\psfig{file=#2,scale=#4}}
\caption{#3}
\label{#1}
\end{figure}}

% Document specific macros go here
\DeclareMathOperator{\Gal}{Gal}
\DeclareMathOperator{\cl}{cl}
\newcommand\bk{\ensuremath{\mathbf{k}}}
\newcommand\cO{\ensuremath{\mathcal{O}}}
\newcommand\oQ{\ensuremath{\overline{\Q}}}
\newcommand\bbA{\ensuremath{\mathbb{A}}}

\begin{document}

\maketitle

\begin{abstract}
We give a character-theory free proof that for finite groups $G_1$ and $G_2$, the
irreducible representations of $G_1 \times G_2$ over an algebraically closed
field of characteristic $0$ consist of tensor
products of irreducible representations of the $G_i$.  Because our
proof avoids character theory, it generalizes to many other similar situations.
\end{abstract}

The goal of this note is to prove the following theorem.

\begin{maintheorem}
\label{theorem:main}
Let $G$ and $H$ be finite groups and let $\bk$ be an algebraically closed field of
characteristic $0$.  The irreducible representations
of $G \times H$ over $\bk$ are precisely those of the form $V \otimes W$, where
$V$ and $W$ are irreducible representations of $G$ and $H$, respectively.
\end{maintheorem}

\noindent
Here is an example to show that the assumption that $\bk$ is algebraically closed
is necessary:

\begin{example}
Consider the action of $\Z/6$ on $\R^2$ where the generator acts as rotation by
$e^{2\pi i/6}$.  This is an irreducible representation; however, despite the fact
that $\Z/6 \cong \Z/3 \times \Z/2$, it does not decompose into the tensor
product of representations of $\Z/3$ and $\Z/2$.
\end{example}

The standard proof of Theorem \ref{theorem:main} (say, from \cite{SerreBook}) uses character theory.  This
has two downsides:
\begin{compactitem}
\item It is indirect, and gives little insight into why the theorem is true.
\item It does not generalize to other situations, for instance to rational representations
of products of reductive Lie groups.
\end{compactitem}
We will give a direct proof.  The only special fact we will use about finite groups
is Maschke's theorem saying that in characteristic $0$, their representations decompose uniquely
into direct sums of irreducible representations, so this proof applies in many other
situations as well.

\begin{proof}[Proof of Theorem \ref{theorem:main}]
We divide the proof into two parts.

\begin{claim}{1}
Let $V$ (resp.\ $W$) be an irreducible representation of $G$ (resp.\ $H$) over $\bk$.  Then
$V \otimes W$ is an irreducible representation of $G \times H$. 
\end{claim}

By assumption, $V$ and $W$ are simple modules over the group rings
$\bk[G]$ and $\bk[H]$, respectively.  
Since $\bk$ is algebraically closed, we can apply the Jacobson Density Theorem 
(see, e.g.\ \cite{PutmanDensity})
and see that the resulting ring maps
$\phi\colon \bk[G] \rightarrow \End_{\bk}(V)$ and $\psi\colon \bk[H] \rightarrow \End_{\bk}(W)$ 
are surjections.  It follows that the ring map
\[\bk[G \times H] \cong \bk[G] \otimes \bk[H] \stackrel{\phi \otimes \psi}{\longrightarrow} \End_{\bk}(V) \otimes \End_{\bk}(W) \cong \End_{\bk}(V \otimes W)\]
is a surjection and thus that $V \otimes W$ is a simple $\bk[G \times H]$-module, as desired.

\begin{claim}{2}
Assume that $U$ is an irreducible representation of $G \times H$ over $\bk$.  
Then $U \cong V \otimes W$, where $V$ (resp.\ $W$) is an irreducible representation
of $G$ (resp.\ $H$) over $\bk$.
\end{claim}

First regard $U$ as a representation of $H$.  As such, $U$ decomposes as a direct sum
of $H$-isotypic components.  Since the action of $G$ on $U$ commutes with the action of $H$, it
must preserve these isotypic components.  Since $U$ was assumed to be irreducible, it follows
that $U$ must have a single $H$-isotypic component, i.e.\ that $U \cong W^{\oplus m}$ for some
irreducible $H$-representation $W$ and some $m \geq 0$.  Consider the $\bk$-linear map
\[\Psi\colon \Hom_H(W,U) \otimes W \rightarrow U\]
defined via the formula $\Psi(\rho \otimes \vec{w}) = \rho(\vec{w})$.  
Since $U \cong W^{\oplus m}$, this
map is surjective.  Also, since $\bk$ is algebraically closed we can apply Schur's Lemma to see
that
\[\Hom_H(W,U) = \Hom_H(W,W^{\oplus m}) \cong \bk^m.\]
We deduce that $\Psi$ is a surjective linear map between vector spaces of the same dimension, 
so $\Psi$ is an isomorphism.

The commuting actions of $G$ and $H$ on $U$ thus can be transported via $\Psi$ to give commuting
actions of $G$ and $H$ on $\Hom_H(W,U) \otimes W$.  These actions are easily understood:
\begin{compactitem}
\item The group $H$ acts trivially on $\Hom_H(W,U)$ and acts on $W$ as the given representation.  Indeed, for
$h \in H$ and $\rho \in \Hom_H(W,U)$ and $\vec{w} \in W$ we have
\[h \cdot \Psi(\rho \otimes \vec{w}) = h \cdot \rho(\vec{w}) = \rho(h \cdot \vec{w}) = \Psi(\rho \otimes h \cdot \vec{w}).\]
\item The group $G$ acts on $\Hom_H(W,U)$ by postcomposition (via its action on $U$) and acts trivially on
$W$.  Indeed, for $g \in G$ and $\rho \in \Hom_H(W,U)$ and $\vec{w} \in W$ we have
\[g \cdot \Psi(\rho \otimes \vec{w}) = g \cdot \rho(\vec{w}) = (g \cdot \rho)(\vec{w}) = \Psi(g \cdot \rho \otimes \vec{w}).\]
\end{compactitem}
Since $U$ was assumed to be irreducible, it follows that $V:=\Hom_H(W,U)$ must be an irreducible
$G$-representation.
The decomposition $U \cong V \otimes W$ is precisely the one we claimed must exist.
\end{proof}

\begin{thebibliography}{HH}
\begin{footnotesize}
\setlength{\itemsep}{-1mm}

\bibitem{PutmanDensity}
A. Putman, The Jacobson density theorem, informal note.

\bibitem{SerreBook}
J.-P. Serre, {\it Linear representations of finite groups}, translated from the second French edition by Leonard L. Scott, Springer-Verlag, New York, 1977.

\end{footnotesize}
\end{thebibliography}

\begin{footnotesize}
\noindent
\begin{tabular*}{\linewidth}[t]{@{}p{\linewidth - \widthof{Department of Mathematics}}@{}p{\widthof{Department of Mathematics}}@{}}
&{\raggedright
Andrew Putman\\
Department of Mathematics\\
University of Notre Dame \\
255 Hurley Hall\\
Notre Dame, IN 46556\\
{\tt andyp@nd.edu}}
\end{tabular*}\hfill
\end{footnotesize}


\end{document}

