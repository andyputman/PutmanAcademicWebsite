\documentclass[11pt]{article}

\title{\vspace{-50pt}Algebraicity of matrix entries of representations}
\author{Andrew Putman\vspace{-6pt}}
\date{}

\usepackage{amsmath,amssymb,amsthm,amscd,amsfonts}
\usepackage{mathtools}
\usepackage{epsfig,pinlabel,paralist}
\usepackage[vmargin=1in, hmargin=1.25in]{geometry}
\usepackage[font=small,format=plain,labelfont=bf,up,textfont=it,up]{caption}
\usepackage{type1cm}
\usepackage{calc}
\usepackage{enumerate}
\usepackage{bm}
\usepackage[all,cmtip]{xy}
\usepackage{eucal}
\usepackage{paralist}
\usepackage{lmodern}
\usepackage[T1]{fontenc}
\usepackage{etoolbox}

\usepackage[bookmarks, bookmarksdepth=2, colorlinks=true, linkcolor=blue, citecolor=blue, urlcolor=blue]{hyperref}

\apptocmd{\thebibliography}{\raggedright}{}{}

\newtheorem{stepx}{Step}
\newenvironment{step}[1]
 {\renewcommand\thestepx{#1}\stepx}
 {\endstepx}

\newtheorem{substepx}{Substep}
\newenvironment{substep}[1]
 {\renewcommand\thesubstepx{#1}\substepx}
 {\endsubstepx}

\numberwithin{equation}{section}

\theoremstyle{plain}
\newtheorem{theorem}{Theorem}[section]
\newtheorem*{rochlin}{Rochlin's Theorem}
\newtheorem*{rochlin2}{Rochlin's Theorem'}
\newtheorem{maintheorem}{Theorem}
\newtheorem{proposition}[theorem]{Proposition}
\newtheorem{lemma}[theorem]{Lemma}
\newtheorem*{unnumberedlemma}{Lemma}
\newtheorem{sublemma}[theorem]{Sublemma}
\newtheorem{corollary}[theorem]{Corollary}
\newtheorem{conjecture}[theorem]{Conjecture}
\newtheorem{question}[theorem]{Question}
\newtheorem{problem}[theorem]{Problem}
\renewcommand{\themaintheorem}{\Alph{maintheorem}}

\theoremstyle{definition}
\newtheorem*{example}{Example}
\newtheorem{assumption}{Assumption}
\newtheorem{defn}[theorem]{Definition}
\newenvironment{definition}[1][]{\begin{defn}[#1]\pushQED{\qed}}{\popQED \end{defn}}
\newtheorem{notation}[theorem]{Notation}

\newtheorem*{remark}{Remark}

% Sets of Functions
\DeclareMathOperator{\Hom}{Hom}
\DeclareMathOperator{\Iso}{Iso}
\DeclareMathOperator{\Diff}{Diff}
\DeclareMathOperator{\Homeo}{Homeo}
\DeclareMathOperator{\Ker}{ker}
\DeclareMathOperator{\Coker}{coker}
\DeclareMathOperator{\Image}{Im}

% My Favorite Groups
\DeclareMathOperator{\Mod}{Mod}
\DeclareMathOperator{\MCG}{MCG}
\newcommand\Torelli{\ensuremath{{\mathcal I}}}
\DeclareMathOperator{\IA}{IA}
\DeclareMathOperator{\Sp}{Sp}
\DeclareMathOperator{\GL}{GL}
\DeclareMathOperator{\SL}{SL}
\DeclareMathOperator{\SO}{SO}
\DeclareMathOperator{\PSL}{PSL}
\DeclareMathOperator{\U}{U}
\DeclareMathOperator{\SU}{SU}
\newcommand\SLLie{\ensuremath{\mathfrak{sl}}}
\newcommand\SpLie{\ensuremath{\mathfrak{sp}}}
\newcommand\GLLie{\ensuremath{\mathfrak{gl}}}
\DeclareMathOperator{\Mat}{Mat}
\DeclareMathOperator{\End}{End}

% Important Spaces
\newcommand\HBolic{\ensuremath{\mathbb{H}}}
\newcommand\Teich{\ensuremath{{\mathcal T}}}
\newcommand\CNosep{\ensuremath{\mathcal{CNS}}}
\newcommand\Sphere[1]{\ensuremath{\mathbf{S}^{#1}}}
\newcommand\Ball[1]{\ensuremath{\mathbf{B}^{#1}}}
\newcommand\Proj{\ensuremath{\mathbb{P}}}
\DeclareMathOperator{\Gr}{Gr}

% Number Systems
\newcommand\R{\ensuremath{\mathbb{R}}}
\newcommand\C{\ensuremath{\mathbb{C}}}
\newcommand\Z{\ensuremath{\mathbb{Z}}}
\newcommand\Q{\ensuremath{\mathbb{Q}}}
\newcommand\N{\ensuremath{\mathbb{N}}}
\newcommand\Field{\ensuremath{\mathbb{F}}}
\DeclareMathOperator{\Char}{char}

% (Co-)Homology
\DeclareMathOperator{\HH}{H}
\newcommand\RH{\ensuremath{\widetilde{\HH}}}
\newcommand\Chain{\ensuremath{{\rm C}}}

% Misc
\DeclareMathOperator{\Max}{max}
\DeclareMathOperator{\Min}{min}
\DeclareMathOperator{\Th}{th}
\DeclareMathOperator{\Aut}{Aut}
\DeclareMathOperator{\Out}{Out}
\DeclareMathOperator{\Interior}{Int}
\newcommand\Span[1]{\ensuremath{\langle #1 \rangle}}
\newcommand\CaptionSpace{\hspace{0.2in}}
\DeclareMathOperator{\Dim}{dim}
\newcommand\Set[2]{\ensuremath{\{\text{#1 $|$ #2}\}}}

% Figures
\newcommand\Figure[4]{
\begin{figure}[t]
\centering
\centerline{\psfig{file=#2,scale=#4}}
\caption{#3}
\label{#1}
\end{figure}}

% Document specific macros go here
\DeclareMathOperator{\Gal}{Gal}
\DeclareMathOperator{\cl}{cl}
\newcommand\bk{\ensuremath{\mathbf{k}}}
\newcommand\cO{\ensuremath{\mathcal{O}}}
\newcommand\oQ{\ensuremath{\overline{\Q}}}
\newcommand\bbA{\ensuremath{\mathbb{A}}}

\begin{document}

\maketitle

\begin{abstract}
We prove that with respect to an appropriate basis, the matrices associated
to complex representations of finite groups have entries lying in algebraic
number rings.  
\end{abstract}

Fix a finite group $G$.  All representations of $G$ discussed in this note are
finite dimensional.  Given an $n$-dimensional $\C$-representation $V$ of $G$, we will
say that $V$ is defined over a subring $R$ of $\C$ if there is an
$R[G]$-module $V'$ with the following two properties:
\begin{compactitem}
\item $V'$ is a rank-$n$ free $R$-module.
\item There is an isomorphism $V' \otimes_R \C \cong V$ of $\C[G]$-modules.
\end{compactitem}
A more pedestrian way of saying this is that with respect an appropriate
basis for $V$, the matrices representing the action of $G$ on $V$
have entries in $\GL_n(R) \subset \GL_n(\C)$.

Recall that a number ring $\cO_{\bk}$ is the ring of integers in an algebraic
number field $\bk$, i.e.\ a finite extension of $\Q$.  The goal of this note
is to prove the following standard theorem.

\begin{maintheorem}
\label{theorem:algebraicnumber}
For all $\C$-representations $V$ of $G$, there exists a number ring
$\cO_{\bk}$ such that $V$ is defined over $\cO_{\bk}$.
\end{maintheorem}
\begin{proof}
The proof will have two steps.

\begin{step}{1}
There exists an algebraic number field $\bk$ such that $V$ is defined
over $\bk$.
\end{step}

Since $G$ is finite, it is enough to show that $V$ is defined over $\oQ$.
Without loss of generality, $V$ is irreducible.  This implies that
$V$ is isomorphic to a subrepresentation of the regular representation $\C[G]$.
The regular representation $\oQ[G]$ over $\oQ$ can be decomposed as a direct
sum of irreducible representations, and the only thing that might go wrong
is that this decomposition might not be fine enough, i.e.\
there is an irreducible subrepresentation $W$ of $\oQ[G]$ such that
$W \otimes \C$ is reducible and $V$ is isomorphic to a proper
subrepresentation of $W \otimes \C$.

Assume that this happens.  Let $n = \dim(V)$, and define
\[X = \Set{$U \in \Gr_n(W)$}{$g(U) = U$ for all $g \in G$}.\]
Thus $X$ is a closed subvariety of the Grassmannian $\Gr_n(W)$, which is a projective variety
defined over $\oQ$.  Our assumption that $W$ is irreducible implies
that $X(\oQ) = \emptyset$, while our assumption that $W \otimes \C$ is reducible
and $V$ is isomorphic to a proper subrepresentation of $W \otimes \C$ implies
that $X(\C) \neq \emptyset$.

This is impossible.  Indeed, if $Y \subset \bbA^m_{\oQ}$ is an open affine
subset of $X$ defined by a radical ideal $I \subset \oQ[x_0,\ldots,x_m]$, then 
since $Y(\oQ) = \emptyset$ the
Nullstellensatz says that $I = \oQ[x_0,\ldots,x_m]$.  This is preserved when we
extend scalars to $\C$, i.e.\ we have
\[Y(\C) = V(I \otimes \C) = V(\C[x_0,\ldots,x_m]) = \emptyset.\]
The desired result follows.

\begin{step}{2}
There exists a finite extension $\bk'$ of $\bk$ such that $V$ is defined
over $\cO_{\bk'}$.
\end{step}

By the previous step, there exists an action of $G$ on $\bk^n$ such that
$V \cong \bk^n \otimes \C$ as $\C[G]$-modules.  Let $L \subset \bk^n$ be
$\cO_{\bk}$-submodule spanned by the $G$-orbit of the standard lattice
$\cO_{\bk}^n \subset \bk^n$, so $L$ is a finitely generated $\cO_{\bk}$-submodule
of $\bk^n$ that is preserved by $G$ 
such that $L \otimes \bk = \bk^n$.  If we had an $\cO_{\bk}$-module isomorphism
$L \cong \cO_{\bk}^n$, then we would be done.  Unfortunately, this need not be true;
however, the classification of finitely generated modules over a Dedekind domain
shows that
\[L \cong I_1 \oplus \cdots \oplus I_n\]
for some nonzero ideals $I_1,\ldots,I_n  \subset \cO_{\bk}$.

The problem is that the $I_i$ might not be principal ideals, i.e.\ they might
define nonzero elements $[I_i]$ of the class group $\cl(\cO_{\bk})$.  The
class group is a finite abelian group, so we can find
$k_1,\ldots,k_n \geq 1$ such that $k_i [I_i] = 0$.  Pick the $k_i$ to
be the minimal such integers.  By definition, the ideal $I_i^{k_i}$ is
principal, i.e.\ there exists some $a_i \in \cO_{\bk}$ such that
\[I_i^{k_i} = (a_i).\]
Define
\[\bk' = \bk[a_1^{1/k_1},\ldots,a_n^{1/k_n}],\]
so
\[I_i \otimes_{\cO_{\bk}} \cO_{\bk'} = (a_i^{1/k_i}) \subset \cO_{\bk'}.\]
Setting 
\[L' = L \otimes_{\cO_{\bk}} \cO_{\bk'} \subset (\bk')^n,\]
we then have
\[L' \cong (I_1 \otimes \cO_{\bk'}) \oplus \cdots \oplus (I_n \otimes \cO_{\bk'}) \cong \cO_{\bk'}^n.\]
We conclude that our original representation is defined over $\cO_{\bk'}$, as desired.
\end{proof}

\begin{footnotesize}
\noindent
\begin{tabular*}{\linewidth}[t]{@{}p{\linewidth - \widthof{Department of Mathematics}}@{}p{\widthof{Department of Mathematics}}@{}}
&{\raggedright
Andrew Putman\\
Department of Mathematics\\
University of Notre Dame \\
255 Hurley Hall\\
Notre Dame, IN 46556\\
{\tt andyp@nd.edu}}
\end{tabular*}\hfill
\end{footnotesize}


\end{document}

